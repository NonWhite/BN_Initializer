\section{Experiments and Results}
\label{sec:experiments}

After implementing all the approaches in section~\ref{subsec:solutions}, some experiments were done in order to compare them. First, we show the setup for all experiments and then show the final results.

\subsection{Experiments Setup}
\label{subsec:configuration}
	Table~\ref{tab:datasets} shows the characteristics of each data set used in the experiments.
	\begin{table}[ h ]
		\centering
		\begin{tabular}{ | l | c | c | c | }
			\hline
			Dataset & n (\#attributes) & N (\#instances) & Avg. \#vals/var\\ \hline
			Census & 13 & 45000 & 7.5\\ \hline
			Dataset 2 & $n_2$ & $N_2$ & $m_2$ \\ \hline
			Dataset 3 & $n_3$ & $N_3$ & $m_3$\\ \hline
			Dataset 4 & $n_4$ & $N_4$ & $m_4$\\ \hline
		\end{tabular}
		\caption{Data sets characteristics}
		\label{tab:datasets}
	\end{table}
	Also, the values for each general parameter are given below.
	\begin{itemize}
		\item Maximum number of parents ($d$): 4
		\item Number of iterations in Greedy Search ($K$): 100
		\item Scoring function used: Bayesian Criterion Information (BIC)
	\end{itemize}
	Finally, as in subsections~\ref{subsub:fasunweighted} and~\ref{subsub:fasweighted} we have to choose a field as a root for performing the DFS, we have $n$ possible different DAGs and then $n$ possible different orders and use each one as an initial solution for Greedy Search. Also, for the random version, we generate $n$ different random orders.

\subsection{Results}
\label{subsec:results}
	After running the three versions explained in~\ref{subsec:solutions} for each data set, we compare the score for the best network found, the CPU Time to generate the initial solution and the CPU Time for getting the final network. Tables~\ref{tab:costs} and~\ref{tab:times} show them respectively.
	\begin{table}[ h ]
		\centering
		\begin{tabular}{ | l | c | c | c | }
			\hline
			Dataset & Random Sol. & Unweighted Sol. & Weighted Sol.\\ \hline
			Census & -165350 & -14683 & $m_1$ \\ \hline
			Dataset 2 & $n_2$ & $N_2$ & $m_2$ \\ \hline
			Dataset 3 & $n_3$ & $N_3$ & $m_3$ \\ \hline
			Dataset 4 & $n_4$ & $N_4$ & $m_4$ \\ \hline
		\end{tabular}
		\caption{Maximum cost using each approach}
		\label{tab:costs}
	\end{table}
	
	% CPU Time for initialization and time to converge (or total)
	\begin{table}[ h ]
		\centering
		\begin{tabular}{ | l | c | c | c | }
			\hline
			Dataset & Random Sol. & Unweighted Sol. & Weighted Sol.\\ \hline
			Census & 10.2 / 72.14 & 93.51 / 48.89 & $m_1$ \\ \hline
			Dataset 2 & $n_2$ & $N_2$ & $m_2$ \\ \hline
			Dataset 3 & $n_3$ & $N_3$ & $m_3$ \\ \hline
			Dataset 4 & $n_4$ & $N_4$ & $m_4$ \\ \hline
		\end{tabular}
		\caption{CPU Time (in seconds) for initializing and total using each approach}
		\label{tab:times}
	\end{table}
	Finally, Figure~\ref{fig:converge} shows the converge curve for each data set using each approach.
	\begin{figure}[H]
		\centering
		\begin{subfigure}{.4\textwidth}
			\centering
			\includegraphics[height=5cm]{images/census}
			\caption{Census}
			\label{fig:census}
		\end{subfigure}
		\begin{subfigure}{.4\textwidth}
			\centering
			\includegraphics[height=5cm]{images/dataset2}
			\caption{Dataset \#2}
			\label{fig:dataset2}
		\end{subfigure}
		\begin{subfigure}{.4\textwidth}
			\centering
			\includegraphics[height=5cm]{images/dataset3}
			\caption{Dataset \#3}
			\label{fig:dataset3}
		\end{subfigure}
		\begin{subfigure}{.4\textwidth}
			\centering
			\includegraphics[height=5cm]{images/dataset4}
			\caption{Dataset \#4}
			\label{fig:dataset4}
		\end{subfigure}
		\caption{Converge curve for datasets}
		\label{fig:converge}
	\end{figure}