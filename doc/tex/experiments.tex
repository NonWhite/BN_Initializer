\section{Experiments and Results}
\label{sec:experiments}

After implementing all the approaches in section~\ref{subsec:solutions}, some experiments were done in order to compare them. First, we show the setup for all experiments and then show the final results.

\subsection{Experiments Setup}
\label{subsec:configuration}
	Table~\ref{tab:datasets} shows the characteristics of each data set used in the experiments.
	\begin{table}[ h ]
		\centering
		\begin{tabular}{ | l | c | c | }
			\hline
			Dataset & n (\#attributes) & N (\#instances) \\ \hline
			Alarm & 37 & 1000 \\ \hline
			Census & 15 & 45222 \\ \hline
			Epigenetics & 30 & 72228 \\ \hline
			Image & 20 & 2310 \\ \hline
			Letter & 17 & 20000 \\ \hline
			Mushroom & 23 & 8124 \\ \hline
			Sensors & 25 & 5456 \\ \hline
			SteelPlates & 28 & 1941 \\ \hline
		\end{tabular}
		\caption{Data sets characteristics}
		\label{tab:datasets}
	\end{table}
	Also, the values for each general parameter are given below.
	\begin{itemize}
		\item Maximum number of parents ($d$): 3
		\item Number of iterations in Greedy Search ($K$): 100
		\item Scoring function used: Bayesian Criterion Information (BIC)
	\end{itemize}
	Finally, as in subsections~\ref{subsub:fasunweighted} and~\ref{subsub:fasweighted} we have to choose a field as a root for performing the DFS, we have $n$ possible different DAGs and then $n$ possible different orders and use each one as an initial solution for Greedy Search. Also, for the random version, we generate $n$ different random orders.

\subsection{Results}
\label{subsec:results}
	After running the three versions explained in~\ref{subsec:solutions} for each data set, we compare the score for the best network found, the percentage of solutions that converge to the best score and the average number of iterations. The results are showed in Table~\ref{tab:comparison}.
	
	% Best Score, %Best solutions, Avg. Iterations,
	\begin{table}[ h ]
		\centering
		\begin{tabular}{ | l | c | c | c | c | c | c | c | c | c | }
			\hline
			Dataset & \multicolumn{3}{|c|}{Random Sol.} & \multicolumn{3}{|c|}{Unweighted Sol.} & \multicolumn{3}{|c|}{Weighted Sol.} \\ \cline{2-10}
			& B. Sc. & \%B. Sol. & Avg.It. & B. Sc. & \%B. Sol. & Avg.It. & B. Sc. & \%B. Sol. & Avg.It. \\ \hline
			Alarm & 14692.868 & 2.70 & 7.67 & 14692.871 & 2.70 & 7.35 & 14692.874 & 2.70 & 7.54 \\ \hline
			Census & 569131.706 & 42.86 & 4.29 & 569131.706 & 7.14 & 4.71 & 569131.706 & 21.43 & 5.5 \\ \hline
			Epigenetics & 214767.676 & 86.67 & 2.67 & 214767.676 & 83.33 & 3.67 & 214767.676 & 80.00 & 4.9 \\ \hline
			Image & 23386.372 & 55.00 & 4.85 & 23386.37 & 30.00 & 5.70 & 23386.37 & 55.00 & 4.60 \\ \hline
			Letter & 161941.091 & 88.23 & 3.71 & 161941.091 & 82.35 & 3.9 & 161941.091 & 88.23 & 3.17 \\ \hline
			Mushroom & 79895.412 & 69.56 & 4.61 & 79895.412 & 91.30 & 3.83 & 79895.412 & 30.43 & 2.52 \\ \hline
			Sensors & 80202.257 & 4.00 & 6.76 & 80200.719 & 4.00 & 10.24 & 80200.378 & 4.00 & 8.52 \\ \hline
			SteelPlates & 24652.807 & 82.14 & 3.53 & 24652.807 & 64.29 & 6.14 & 24652.807 & 85.71 & 5.57 \\ \hline
		\end{tabular}
		\caption{Best score obtained (B. Sc.), Percentage of solutions that converge to best score (\% B. Sol.) and Average number of iterations (Avg. It.) using each approach}
		\label{tab:comparison}
	\end{table}
	
	Finally, Figure~\ref{fig:converge} shows the converge curve for each data set using each approach.
	\begin{figure}[H]
		\centering
		\begin{subfigure}{.3\textwidth}
			\centering
			\includegraphics[height=3.5cm]{images/alarm}
			\caption{Alarm}
			\label{fig:alarm}
		\end{subfigure}
		\begin{subfigure}{.3\textwidth}
			\centering
			\includegraphics[height=3.5cm]{images/census}
			\caption{Census}
			\label{fig:census}
		\end{subfigure}
		\begin{subfigure}{.3\textwidth}
			\centering
			\includegraphics[height=3.5cm]{images/epigenetics}
			\caption{Epigenetics}
			\label{fig:epigenetics}
		\end{subfigure}
		\begin{subfigure}{.3\textwidth}
			\centering
			\includegraphics[height=3.5cm]{images/image}
			\caption{image}
			\label{fig:image}
		\end{subfigure}
		\begin{subfigure}{.3\textwidth}
			\centering
			\includegraphics[height=3.5cm]{images/letter}
			\caption{Letter}
			\label{fig:letter}
		\end{subfigure}
		\begin{subfigure}{.3\textwidth}
			\centering
			\includegraphics[height=3.5cm]{images/mushroom}
			\caption{Mushroom}
			\label{fig:mushroom}
		\end{subfigure}
		\begin{subfigure}{.3\textwidth}
			\centering
			\includegraphics[height=3.5cm]{images/sensors}
			\caption{Sensors}
			\label{fig:sensors}
		\end{subfigure}
		\begin{subfigure}{.3\textwidth}
			\centering
			\includegraphics[height=3.5cm]{images/steelPlates}
			\caption{SteelPlates}
			\label{fig:steelplates}
		\end{subfigure}
		\caption{Converge curve for datasets}
		\label{fig:converge}
	\end{figure}