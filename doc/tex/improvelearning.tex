\section{Generating informed initial solutions}
\label{sec:improve}

From section~\ref{subsec:greedysearch}, we already know the Greedy Search algorithm and the more popular approaches using it. In all of the approaches showed can be noticed that a very important step in the algorithm is in the first line, where an initial solution is generated because if we have a good one, the solution will converge faster. We focus in the Order-based one ((in subsection~\ref{subsub:orderbased}) . At the start of this section, the most common method will be explained. After that, we explain every new proposed method to generate an initial solution for Order-based Greedy Algorithm and the Feedback Arc Set (FAS) Problem.

\subsection{Random solution}
\label{subsec:randomapproach}
	This is the most common and easy-to-implement approach used. An easy way to shuffle a list $L$ with size $n$ is the Fisher-Yates algorithm \cite{FisherYates98}.

\subsection{DFS-based solution}
\label{subsec:dfsapproach}
	Generating random solutions means having a huge search space ($n!$, where $n$ is the number of attributes in the dataset). In order to reduce this search space we can use a graph $G^*$ (in Figure~\ref{fig:example} that tell us some information about the relationships between the attributes.
	\begin{figure}[H]
		\centering
		\subsection{Examples}

\begin{frame}[fragile]
	\begin{figure}
		\centering
		\begin{tikzpicture}[ scale = 0.6 ]
	\tikzset{ vertex/.style = { shape = rectangle , draw , minimum size = 2em , rounded corners } }
	\tikzset{ edge/.style = { ->,> = latex' } }
	% vertices
	\node[ vertex ] (B) at  ( 4 , 4 ) { ${Buying}$ } ;
	\node[ vertex ] (D) at  ( 2 , 2 ) { ${Doors}$ } ;
	\node[ vertex ] (P) at  ( 6 , 2 ) { ${Persons}$ } ;
	\node[ vertex ] (M) at  ( 0 , 0 ) { ${Mantain}$ } ;
	\node[ vertex ] (S) at  ( 4 , 0 ) { ${Safety}$ } ;
	\node[ vertex ] (L) at  ( 8 , 0 ) { ${Luggage}$ } ;

	%edges
	\draw[ edge ] (B) to (D) ;
	\draw[ edge ] (B) to (P) ;
	\draw[ edge ] (D) to (M) ;
	\draw[ edge ] (D) to (S) ;
	\draw[ edge ] (P) to (S) ;
	\draw[ edge ] (P) to (L) ;
\end{tikzpicture}
	\end{figure}
	\begin{block}{}
		$\P( B , M , D , P , L , S ) = \P( B ) \P( D \mid B ) \P( P \mid B ) \P( M \mid D ) \P( S \mid D , P ) \P( L \mid P )$
	\end{block}
	This requires $4 + (4 \times 4) + (3 \times 4) + (4 \times 4) + (3 \times 4 \times 3) + (3 \times 3) = 93$ probabilities instead of $1728$
\end{frame}
	
\begin{frame}[fragile]
	\begin{columns}
		\begin{column}{.3\linewidth}
			Consider each variable has $k$ values:\\
			We requires $k^{33}$ probabilities without independences.
		\end{column}
		\begin{column}{.6\linewidth}
			\begin{figure}
				\centering
				\includegraphics[height=17em]{images/complexbn}
			\end{figure}
		\end{column}
	\end{columns}
\end{frame}
		\caption{Graph $G^*$ built using equation~\ref{eq:bestparents} }
		\label{fig:example}
	\end{figure}
	From $G^*$, we know that some orders does not generate some of the parent sets showed there. For instance, we could have attribute $C$ before $D$, or $A$ before $B$,$C$ or $D$ in the order generated. Having in consideration all these relations we have the following possible orders:
	\begin{multicols}{7}
		\begin{verbatim}
			ABDC
			ACDB
			ADBC
			ADCB
			BDAC
			BDCA
			CDAB
			CDBA
			DABC
			DACB
			DBAC
			DBCA
			DCAB
			DCBA
		\end{verbatim}
	\end{multicols}
	As can be noticed we have 14 possible orders given $G^*$, but $4! = 24$ possible orders (the number of permutations using 4 nodes) using a random approach. So we reduce the search space and also improve the possibilities that getting a good initial order because the relations are the best parent sets for each attribute. Also, the difference could be greater in problems with more attributes.
	Taking into consideration the previous analysis, we propose the following algorithm to generate initial solutions:
	\begin{enumerate}
		\item Find the best parent set ${Pa}( X_i )$ for each attribute $X_i$
		\item Build a graph $G^*$ using the relations between each attribute $X_i$ and ${Pa}( X_i )$ as edges
		\item Start with an empty list $L$
		\item While there are nodes not visited do
		\begin{enumerate}
			\item Choose a random node $X_k$ not visited
			\item Perform a DFS on graph $G^*$ using $X_k$ as root and add each node $X_v$ visited to the order $L$
		\end{enumerate}
		\item Return $L$
	\end{enumerate}
	
\subsection{FAS-based solution}
\label{subsec:fasapproach}

In this subsection, we first explained the generic problem and one of the variants of the Feedback Arc Set (FAS) Problem. After that, a reduction of the learning structure bayesian network problem will be explained in order to propose a new algorithm for generating initial solutions.

\subsubsection{Feedback Arc Set Problem}
\label{subsub:fasp}

The generic problem is stated as: Given a graph $G^* = ( V , E )$, a set $F \subseteq E$ is called the Feedback Arc Set (FAS) if every cycle of $G^*$ has at least one edge in $F$. In other words, $F$ is the edge set that if you put if off the graph $G^*$, it becomes a directed acyclic graph (DAG).\\
There are some variants of this problem, but we only focus on finding the minimum cost FAS $F$ from a weighted directed graph $G^*$ defined in equation~\ref{eq:mincostfas}.
\begin{equation}
	\label{eq:mincostfas}
	F_{G^*} = \min_{G^* becomes DAG} \sum_{(u,v) \in E} W_{uv} ,
\end{equation}
where $W_{uv}$ is the weight of the edge from $u$ to $v$ on $G^*$. Although this variant becomes NP-hard, there are some approximation algorithms like the one that is below.
\begin{enumerate}
	\item Start with $F = \emptyset$
	\item While there is a cycle $C$ in $G$ do
	\begin{enumerate}
		\item Find the lowest weight $W_{min}$ of all the edges $(u,v) \in C$
		\item For each edge $(u,v) \in C$, do $W_{uv} = W_{uv} - W_{min}$
		\item If some $W_{uv} = 0$ for some edge $(u,v) \in C$, then add $(u,v)$ to $F$
	\end{enumerate}
	\item For each edge $(u,v) \in F$, add it to $G$ only if it does not build a cycle on the graph
	\item Return $F$
\end{enumerate}

\subsubsection{Algorithm for initial solutions}
\label{subsub:fasalgorithm}
	First, we have the following relation for the problem:
		\begin{equation}
			\label{eq:reduction}
			G = \max_{\{ {Pa}(X_i)\}_i, G \text{ is acyclic}} \sum_{i=1}^n {sc}( X_i , {Pa}( X_i ) ) \leq G^* = \sum_{i=0}^{n} \max {sc}( X_i , {Pa}( X_i ) ) ,
		\end{equation}
	The left part is the original definition of the problem for learning structure of bayesian networks (given in equation~\ref{eq:decomposability}) and the right part is the graph $G^*$ showed in equation~\ref{eq:bestparents}. In other words, the bayesian network $G$ could be an acyclic subgraph of $G^*$. 
	In order to prove that the relation is true, we need to define the weight edges as follows:
		\begin{equation}
			\label{eq:weight}
			W_{ji} = {sc}( X_i , P ) - {sc}( X_i , P \setminus \{ X_j \} ) ,
		\end{equation}
	where $W_{ji}$ is the weight of the edge that goes from $X_j$ to $X_i$ and $P$ is the best parents set for $X_i$. This formula represents the cost of getting $X_j$ out of the set $P$ and it is always a positive number. When the value is so small, it means the parent $X_j$ could be more important. On the contrary if the weight is so big, this means that $X_j$ is not so important in $P$.\\
	Then we can prove that the problem can be reduce with the following equations:
	\begin{equation}
		\label{eq:prove1}
		\min \sum_{i=0}^{n} W_{uv} = \min \sum_{i=0}^{n} ( {sc}( X_i , P ) - {sc}( X_i , P \setminus X_j ) )
	\end{equation}
	\begin{equation}
		\label{eq:prove2}
		\min \sum_{i=0}^{n} {sc}( X_i , P ) - \min \sum_{i=0}^{n} {sc}( X_i , P \setminus X_j )
	\end{equation}
	\begin{equation}
		\label{eq:prove3}
		\max \sum_{i=0}^{n} {sc}( X_i , P ) + \min \sum_{i=0}^{n} {sc}( X_i , P \setminus X_j )
	\end{equation}
	\begin{equation}
		\label{eq:prove4}
		{sc}( G ) + \min \sum_{i=0}^{n} {sc}( X_i , P \setminus X_j ) \approx {sc}( G^* )
	\end{equation}
	This means that the score of DAG $G$ plus the score of the deleted arcs of $G^*$ give an approximation, and also that $G$ could be a subgraph of $G^*$.
	Finally, as we already know how to become $G^*$ to a DAG $G$ using the FAS solution given in~\ref{subsub:fasp}, we propose the new algorithm for getting an initial solution using it as follows:
	\begin{enumerate}
		\item Find the best parent set ${Pa}( X_i )$ for each attribute $X_i$
		\item Build the graph $G^*$ using all the relationships $X_j \implies X_i, X_j \in {Pa}( X_i )$ and put its score as $W_{ji}$
		\item Find the minimum cost FAS $F$
		\item Delete all edges $(u,v) \in F$ from $G^*$ to obtain a network $G$
		\item Return a topological order of $G$
	\end{enumerate}

These new methods for generating initial solutions will be used in next section to learn Bayesian networks with multiple data sets.