\section{Conclusions and Future Work}
\label{sec:conclusions}

Learning Bayesian networks from data is a notably difficult problem, and practitioners often resort to approximate solutions such as greedy search. The quality of the solutions produced by greedy approaches strongly depends on the initial solution. In this work, we proposed two new heuristics for producing topological orderings to be fed into Order-Based Greedy Bayesian network Structure Search methods. One is based on a Depth-First Search traversal of the (cyclic) graph obtained by greedily selecting the best parents for each variable; the other is based on finding an acyclic subgraph of that same graph by solving a related minimum cost Feedback-Arc Set problem. Experiments with real-world datasets containing from 15 to 57 variables demonstrate that compared to the commonly used strategy of generating initial ordering uniformly at random the proposed heuristics lead to better solutions on average, and increase the  convergence of the search with only a small overhead . Although the gains observed in our experiments are small, we expect larger differences for datasets with more variables. A follow-up work should verify this hypothesis.

Our proposed techniques could be adapted to generate initial solutions
also for Structure- and Equivalence-based local search methods by returning  directed acyclic graphs instead of node orderings. Another extension of this work is to employ the proposed heuristics in branch-and-bound solvers such as~\cite{Cassio11} for finding optimal solutions. These ideas are left as future work.
%\vspace{-4mm}
