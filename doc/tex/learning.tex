\section{Aprendizado de redes bayesianas}
\label{sec:aprendizado}

O Algoritmo~\ref{code:original} mostra um pseudocódigo da versão original do método de aprendizado de redes bayesianas.

\begin{lstlisting}[ caption = Busca local no espaço de ordenamentos , label = code:original ]
	L = Permutation( x[ 0 ] , x[ 1 ] , ... , x[ n ] )
	while k < NUM_RESTARTS :
		best = find_order( L )
		if score( best ) > score( L ) :
			L = best
\end{lstlisting}

Além disso, a função ${find\_order}$ é mostrada a continuação. O método ${swap}$ troca os valores entre as posições $i$ e ${i+1}$ de $L$.

\begin{lstlisting}[ caption = Find\_Order , label = code:findorder ]
	best_i = 0
	best_diff = -INF
	score_orig = score( L )
	for i in range( n - 1 ) :
		swap( L , i )
		d_i = score( L ) - score_orig
		if d_i > best_diff :
			best_diff = d_i
			best_i = i
		swap( L , i )
	swap( L , best_i )
	return L
\end{lstlisting}

Na seção~\ref{sec:experimentos} este algoritmo será usado para encontrar redes bayesianas para vários conjuntos de dados.