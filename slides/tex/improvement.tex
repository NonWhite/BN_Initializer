\section{Initializing Heuristics}

\subsection{Initial solution}
	\begin{frame}
		The selection of a good initial solution is crucial for avoiding convergence to poor local maxima in Order-Based Learning. Traditionally, this is attempted by randomly generating initial solutions (i.e., a node ordering) in order to cover as much as possible of the search space. But this also means that we have $n!$ possible orders and this is a huge disadvantage for problems with lots of variables.\\
		We propose two different approaches to reduce the space of possible orders:
		\begin{itemize}
			\item DFS-based approach
			\item FAS-based approach
		\end{itemize}
	\end{frame}

\subsection{Upper bound}
	\begin{frame}
		We can have an upper bound for ${sc}( G^* )$ by getting ${sc}( \overline G )$ where
			\[ \overline{G} = \arg \sum_i \max_{{Pa}(X_i)} sc( X_i, {Pa}( X_i ) ) \]
		is the directed graph where the parents ${Pa}( X_i )$ of each node $X_i$ are selected as to maximize the local score ${sc}( X_i , {Pa}( X_i ) )$. We call the parents of a variable in $\overline G$ the \alert{best parent set}. Note that $\overline G$ usually contains cycles and it is thus not a solution to the main problem.
	\end{frame}
	
\subsection{DFS-based approach}
	\begin{frame}
		\begin{itemize}
			\item We can exploit the information provided by $\overline G$ and avoid generating orders which are guaranteed sub-optimal
			\item Assume \alert{best parent set} is unique
			\item Consider two variables $X_i$ and $X_j$ in $\overline G$, where $X_j$ is parent of $X_i$, but there is no arc from $X_i$ to $X_j$
			\item No optimal ordering can have $X_i$ preceding $X_j$
		\end{itemize}
		The number of these orderings can be much smaller than the full space of orderings
	\end{frame}
	\begin{frame}{Example}
		\begin{columns}
			\begin{column}{.48\textwidth}
				\begin{figure}
					\centering
					\input{networks/overlineg}
				\end{figure}
				\centering
				Graph $\overline G$
			\end{column}
			\begin{column}{.48\textwidth}
				\begin{figure}
					\centering
					\includegraphics{images/dfsorders}
				\end{figure}
			\end{column}
		\end{columns}
	\end{frame}
	\begin{frame}
		\begin{block}{The algorithm}
			\begin{itemize}
				\item Take as input $\overline G$ and mark all nodes unvisited
				\item Start with an empty list $L$
				\item While there is an unvisited node
				\begin{itemize}
					\item Select an unvisited $X_i$ uniformly random
					\item Perform a depth-first search (DFS) rooted at $X_i$ and add to $L$ the visited nodes
				\end{itemize}
				\item Return $L$
			\end{itemize}
		\end{block}
	\end{frame}

\subsection{FAS-based approach}
	\begin{frame}
		\begin{block}{Disadvantage of DFS approach}
			This approach can be seen as removing edges from $\overline G$ such as to make it a DAG and then extract a topological order. But not all edges are equally relevant in terms of avoiding poor local maxima.
		\end{block}		
	\end{frame}
	\begin{frame}
		\begin{block}{Estimating the relevance}
			We can estimate the relevance of an edge $X_j \rightarrow X_i$ by
			\[ W_{ji} = {sc}( X_i , {Pa}^*( X_i ) ) - {sc}( X_i , {Pa}^*( X_i ) \setminus \{ X_j \} ) \]
			where ${Pa}^*( X_i )$ represents the \alert{best parent set} for $X_i$.\\
			The weight $W_{ji}$ represents the cost of removing $X_j$ from the set ${Pa}^*( X_i )$ and it is always a positive number.\\
			Small values mean that parent $X_j$ is not very relevant to $X_i$, while large values denotes the opposite.
		\end{block}
		We then wish to find a topological ordering of $\overline G$ that violates the least cost of edges.
	\end{frame}
	\begin{frame}
		\begin{block}{Feedback Arc Set}
			\begin{itemize}
				\item Given a directed graph $G = ( V , E )$, a set $F \subseteq E$ is called a Feedback Arc Set (FAS) if every (directed) cycle of $G$ contains at least one edge in $F$.\\
				\item In other words, $F$ is an edge set that if removed makes the graph $G$ acyclic.\\
				\item This could be reduced to the following equation:
					\[ F = \min_{G-F \text{ is a DAG}} \sum_{X_j \rightarrow X_j \in E} W_{ij} \]
			\end{itemize}
		\end{block}
	\end{frame}
	\begin{frame}{Finding FAS $F$}
		\defverbatim
\fasapprox{
\begin{lstlisting}
	MinimumCostFAS( Graph |$G$| ) : Return FAS |$F$|
		|$F$| = empty set
		While there is a cycle |$C$| on |$G$| do
			|$W_{min}$| = lowest weight of all edges in |$C$|
			For each edge |$(u,v) \in C$| do
				|$W_{uv} = W_{uv} - W_{min}$|
			If |$W_{uv} = 0$| add to |$F$|
		For each edge in |$F$|, add it to |$G$| if does not build a cycle
		Return |$F$|
\end{lstlisting}
}
\fasapprox
	\end{frame}
	\begin{frame}
		\begin{block}{The algorithm}
			\begin{itemize}
				\item Take the weighted graph $\overline G$ with weights $W_{ij}$ as input
				\item Find minimum-cost FAS $F$
				\item Remove the edges in $F$ from $\overline G$
				\item Return a topological order from $\overline G - F$
			\end{itemize}
		\end{block}
	\end{frame}