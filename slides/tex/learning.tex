\section{Learning BN}

\subsection{BN Structure Learning Approaches}
	\begin{frame}
		\begin{block}{Constraint-based approaches}
			Perform multiple conditional independence hypothesis testing in order to build a DAG
		\end{block}
		\vskip2em
		\begin{block}{Score-based approaches}
			Associate every DAG with a polynomial-time computable score value and search for structure with high score values
		\end{block}
	\end{frame}

\subsection{Score-based Structure Learning}
	\begin{frame}
		\begin{block}{Learning as optimization}
			Given dataset $D$, select $G$ that maximizes \alert{decomposable} score function ${sc}( G , D )$
		\end{block}
		\begin{itemize}
			\item Score ${sc}( G , D )$ is usually a mix of data fitness $F$ and model complexity $P$:
				\[ {sc}( G , D ) = F( G ) + \psi( N ) \times P( G ) \]
				with $\psi( N ) \geq 0$ is a function of data size $|D| = N$
			\item We usually ommit dependence on $D$: ${sc}( G)$
		\end{itemize}
	\end{frame}
	\begin{frame}
		\begin{block}{Learning as optimization}
			Select $G$ that maximizes \alert{decomposable score function}
			\[ G^* = arg \max_{G: G \text{ is a DAG}} {sc}( G ) \]
			\[ G^* = arg \max_G \sum_i {sc}( X_i , {Pa}( X_i ) ) \]
		\end{block}
	\end{frame}
	
\subsection{Greedy Search Approach}
	\begin{frame}[fragile]
		Greedy Search is a popular approach to find an approximate solution. It relies on the definition of a neighborhood space among solutions and on local moves that search for improving solution in the neighborhood of an incumbent solution
		\vskip1.5em
		\subsection{Greedy Search Approach}
	\begin{frame}[fragile]
		Greedy Search is a popular approach to find an approximate solution
		\vskip1.5em
		\subsection{Greedy Search Approach}
	\begin{frame}[fragile]
		Greedy Search is a popular approach to find an approximate solution
		\vskip1.5em
		\subsection{Greedy Search Approach}
	\begin{frame}[fragile]
		Greedy Search is a popular approach to find an approximate solution
		\vskip1.5em
		\input{algorithm/greedysearch}
	\end{frame}
	\begin{frame}
		Different neighborhoods and local moves rise to different methods such as:
		\begin{itemize}
			\item Structure-based
			\item Equivalence-based
			\item Order-based
		\end{itemize}
	\end{frame}
	\end{frame}
	\begin{frame}
		Different neighborhoods and local moves rise to different methods such as:
		\begin{itemize}
			\item Structure-based
			\item Equivalence-based
			\item Order-based
		\end{itemize}
	\end{frame}
	\end{frame}
	\begin{frame}
		Different neighborhoods and local moves rise to different methods such as:
		\begin{itemize}
			\item Structure-based
			\item Equivalence-based
			\item Order-based
		\end{itemize}
	\end{frame}
	\end{frame}
	\begin{frame}
		Different neighborhoods and local moves rise to different methods such as:
		\begin{itemize}
			\item Structure-based
			\item Equivalence-based
			\item Order-based
		\end{itemize}
	\end{frame}
	
\subsection{Order-based Greedy Search}
	\begin{frame}
		Based on the observation that the problem of learning a Bayesian network can be written as
			\[ G^* = \arg \max_{<} \max_{G \text{ consistent with } <} \sum_{i=1}^{n} {sc}( X_i , {Pa}( X_i ) ) \]
			\[ G^* = \arg \max_{<} \sum_{i=1}^{n} \max_{P \subseteq \{ X_j < X_i \}} {sc}( X_i , P ) \]
		which means that if an optimal ordering over the variables is known, an optimal DAG can be found by maximizing the local scores \alert{independently}
	\end{frame}
	\begin{frame}
		\defverbatim
\orderbased{
\begin{lstlisting}
	OrderBasedGreedySearch( Dataset |$D$| , Order |$L_0$| ) :
	return a BN
		|$L = L_0$|
		For a number of iterations |$K$|
			|$current\_sol = L$|
			For each |$i$| = 1 to |$n-1$| do
				|$L_i = swap( L , i , i+1 )$|
				if |$score( L_i ) > score( current\_sol )$|
					|$current\_sol = L_i$|
			if |$score( current\_sol ) > score( L )$| then
				|$L = current\_sol$|
		Return |${network}( L )$|
\end{lstlisting}
}
\orderbased
		where ${swap}( L , i , i + 1 )$ swaps the values $L[ i ]$ and $L[ i + 1 ]$
	\end{frame}