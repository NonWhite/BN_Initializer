\section{Bayesian Network}

\subsection{Markov property}
	\begin{frame}
		\begin{block}{Markov property}
			Given its parents, every variable is conditionally independent from its non-descendants non-parents
		\end{block}
		\begin{block}{Factorization property}
			\[ \P( X_1 , \ldots , X_n ) = \prod_{i=1}^n \P( X_i \mid {Pa}( X_i ) ) \]
		\end{block}
	\end{frame}
	
	\begin{frame}[fragile]
		\begin{figure}
	 		\centering
			\subsection{Examples}

\begin{frame}[fragile]
	\begin{figure}
		\centering
		\begin{tikzpicture}[ scale = 0.6 ]
	\tikzset{ vertex/.style = { shape = rectangle , draw , minimum size = 2em , rounded corners } }
	\tikzset{ edge/.style = { ->,> = latex' } }
	% vertices
	\node[ vertex ] (B) at  ( 4 , 4 ) { ${Buying}$ } ;
	\node[ vertex ] (D) at  ( 2 , 2 ) { ${Doors}$ } ;
	\node[ vertex ] (P) at  ( 6 , 2 ) { ${Persons}$ } ;
	\node[ vertex ] (M) at  ( 0 , 0 ) { ${Mantain}$ } ;
	\node[ vertex ] (S) at  ( 4 , 0 ) { ${Safety}$ } ;
	\node[ vertex ] (L) at  ( 8 , 0 ) { ${Luggage}$ } ;

	%edges
	\draw[ edge ] (B) to (D) ;
	\draw[ edge ] (B) to (P) ;
	\draw[ edge ] (D) to (M) ;
	\draw[ edge ] (D) to (S) ;
	\draw[ edge ] (P) to (S) ;
	\draw[ edge ] (P) to (L) ;
\end{tikzpicture}
	\end{figure}
	\begin{block}{}
		$\P( B , M , D , P , L , S ) = \P( B ) \P( D \mid B ) \P( P \mid B ) \P( M \mid D ) \P( S \mid D , P ) \P( L \mid P )$
	\end{block}
	This requires $4 + (4 \times 4) + (3 \times 4) + (4 \times 4) + (3 \times 4 \times 3) + (3 \times 3) = 93$ probabilities instead of $1728$
\end{frame}
	
\begin{frame}[fragile]
	\begin{columns}
		\begin{column}{.3\linewidth}
			Consider each variable has $k$ values:\\
			We requires $k^{33}$ probabilities without independences.
		\end{column}
		\begin{column}{.6\linewidth}
			\begin{figure}
				\centering
				\includegraphics[height=17em]{images/complexbn}
			\end{figure}
		\end{column}
	\end{columns}
\end{frame}
		\end{figure}
		\begin{block}{}
			The directed acyclic graph (DAG) above has joint probability distribution:
			\[ \P( I , C , S , A ) = \P( I ) \P( C \mid I ) \P( S \mid C , I ) \P( A \mid S , C , I ) \]
			\[ = \P( I ) \P( C ) \P( S \mid C , I ) \P( A \mid S ) \]
		\end{block}
	\end{frame}

\subsection{Definition}
	\begin{frame}
		A Bayesian Network consists of
		\begin{itemize}
			\item A DAG $G$ over a set of variables $X_1 , \ldots , X_n$
			\item \alert{Probability constraints}: $\P( X_i = k \mid {Pa}( X_i ) = j ) = \theta_{ijk}$
		\end{itemize}
		\begin{block}{Joint Probability Distribution}
			There is a unique probability function consistent with a BN:
			\[ \P( X_1 , \ldots , X_n ) = \prod_{i=1}^n \P( X_i \mid {Pa}( X_i ) ) = \prod_{i=1}^n \theta_{ijk} \]
		\end{block}
	\end{frame}

\subsection{Example}
	\begin{frame}
		\begin{itemize}
			\item Age (A): young, adult, old
			\item Gender (G): male, female
			\item Education (E): primary, high school, university
			\item Occupation (O): employee, self-employed
			\item City size (C): big, small
			\item Transport (T): private (car), public (bus, train, etc)
		\end{itemize}
	\end{frame}
	\begin{frame}
		\begin{columns}
			\begin{column}{.3\textwidth}
				\begin{figure}
					\centering
					\input{networks/census}
				\end{figure}
			\end{column}
			\begin{column}{.7\textwidth}
				\begin{itemize}
					\item Education rates have been increasing over years; young people are more likely to have university degrees than old people
					\item Women are more likely to invest in their education than men; women outnumber men in the vast majority of university-level courses
					\item High education levels is key to getting prestigious professions; jobs requiring university degrees are more easily available in big cities
					\item Preferred means of transport depends on occupation and city size
				\end{itemize}
			\end{column}
		\end{columns}
	\end{frame}
	\begin{frame}
		\begin{columns}
			\begin{column}{.3\textwidth}
				\begin{figure}
					\centering
					\input{networks/census}
				\end{figure}
			\end{column}
			\begin{column}{.7\textwidth}
				$\P( A = young ) = 0.3$\\
				$\P( A = adult ) = 0.5$\\
				$\P( A = old ) = 0.2$\\
				$\P( E = high \mid A = young , G = F ) = 0.7$\\
				$\vdots$\\
				$\P( C = small \mid E = high ) = 0.25$
				$\vdots$
			\end{column}
		\end{columns}
	\end{frame}

\subsection{Causal Interpretation}
	\begin{frame}
		An arc $X \rightarrow Y$ can be interpreted as "$X$ causes $Y$"
		\begin{columns}
			\begin{column}{.3\textwidth}
				\begin{figure}
					\centering
					\input{networks/chain}
				\end{figure}
				causal chain
			\end{column}
			\begin{column}{.3\textwidth}
				\begin{figure}
					\centering
					\begin{tikzpicture}
	\tikzset{ vertex/.style = { shape = circle , draw , minimum size = 2em } }
	\tikzset{ edge/.style = { ->,> = latex' } }
	% vertices
	\node[ vertex ] (X) at  ( 0 , 1 ) { ${X}$ } ;
	\node[ vertex ] (Y) at  ( 1 , 0 ) { ${Y}$ } ;
	\node[ vertex ] (Z) at  ( 2 , 1 ) { ${Z}$ } ;

	%edges
	\draw[ edge ] (Y) to (X) ;
	\draw[ edge ] (Y) to (Z) ;
\end{tikzpicture}
				\end{figure}
				common cause
			\end{column}
			\begin{column}{.3\textwidth}
				\begin{figure}
					\centering
					\input{networks/effect}
				\end{figure}
				common effect
			\end{column}
		\end{columns}
		\vskip2em
		Defining and verifying causality is difficult and controversial: We can loosely define $X$ causes $Y$ if $X$ temporarily precedes and direct influences $Y$
	\end{frame}
	
\subsection{Complete probabilistic model}
	\begin{frame}
		\begin{columns}
			\begin{column}{.3\textwidth}
				\begin{figure}
					\centering
					\input{networks/census}
				\end{figure}
			\end{column}
			\begin{column}{.7\textwidth}
				We can query a Bayesian Network about unspecified probabilities
				\begin{itemize}
					\item Are women more likely to prefer public transport over men:\\
						$\P( T = public \mid G = F ) > \P( T = public \mid G )$?
					\item What is the distribution of ages for people who use private means of transport:\\
						$\P( A \mid T = private )$?
				\end{itemize}
			\end{column}
		\end{columns}
	\end{frame}