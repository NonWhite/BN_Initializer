\subsection{FAS-based approach}
	\begin{frame}
		\begin{block}{Disadvantage of DFS approach}
			This approach can be seen as removing edges from $\overline G$ such as to make it a DAG and then extract a topological order. But not all edges are \alert{equally relevant} in terms of avoiding poor local maxima.
		\end{block}		
	\end{frame}
	
	\begin{frame}
		\begin{block}{Estimating the relevance}
			We can estimate the relevance of an edge $X_j \rightarrow X_i$ by
			\[ W_{ji} = {sc}( X_i , {Pa}^*( X_i ) ) - {sc}( X_i , {Pa}^*( X_i ) \setminus \{ X_j \} ) \]
			where ${Pa}^*( X_i )$ represents the \alert{best parent set} for $X_i$.\\
		\end{block}
		We then wish to find a topological ordering of $\overline G$ that violates the least cost of edges.
	\end{frame}
	
	\begin{frame}{Example}
		\begin{columns}
			\begin{column}{.4\textwidth}
				\begin{figure}
					\centering
					\input{graphs/weightedg}
				\end{figure}
				\centering
				Graph $\overline G$
			\end{column}
			\begin{column}{.5\textwidth}
				\begin{itemize}
					\item $C$ is not very relevant as parent to $D$
					\item $B$ is the most relevant parent of $D$
				\end{itemize}
			\end{column}
		\end{columns}
	\end{frame}
	
	\begin{frame}
		\begin{block}{Min-Cost Feedback Arc Set}
			 Given a weighted directed graph $G = ( V , E )$, a set $F \subseteq E$ is called a Min-Cost Feedback Arc Set (min-cost FAS) if every (directed) cycle of $G$ contains at least one edge in $F$ and the sum of weights is minimum.
				\[ F = \min_{G-F \text{ is a DAG}} \sum_{X_i \rightarrow X_j \in E} W_{ij} \]
		\end{block}
	\end{frame}
	
	\begin{frame}{Finding FAS $F$}
		The following algorithm finds an approximate solution:
		\vskip1em
		\defverbatim
\fasapprox{
\begin{lstlisting}
	MinimumCostFAS( Graph |$G$| ) : Return FAS |$F$|
		|$F$| = empty set
		While there is a cycle |$C$| on |$G$| do
			|$W_{min}$| = lowest weight of all edges in |$C$|
			For each edge |$(u,v) \in C$| do
				|$W_{uv} = W_{uv} - W_{min}$|
			If |$W_{uv} = 0$| add to |$F$|
		For each edge in |$F$|, add it to |$G$| if does not build a cycle
		Return |$F$|
\end{lstlisting}
}
\fasapprox
	\end{frame}
	
	\begin{frame}
		\begin{block}{The algorithm}
			\begin{itemize}
				\item Take the weighted graph $\overline G$ with weights $W_{ij}$ as input
				\item Find min-cost FAS $F$
				\item Remove the edges in $F$ from $\overline G$
				\item Return a topological order from $\overline G - F$
			\end{itemize}
		\end{block}
	\end{frame}